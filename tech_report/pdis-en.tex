%% FEUP THESIS STYLE for LaTeX2e
%% how to use feupteses (English version)
%%
%% FEUP, JCL & JCF, 31 July 2012
%%
%% PLEASE send improvements to jlopes at fe.up.pt and to jcf at fe.up.pt
%%

%%========================================
%% Commands: pdflatex tese
%%           bibtex tese
%%           makeindex tese (only if creating an index)
%%           pdflatex tese
%% Alternative:
%%          latexmk -pdf tese.tex
%%========================================

\documentclass[11pt,a4paper,twoside,openright]{report}

%% For iso-8859-1 (latin1), comment next line and uncomment the second line
\usepackage[utf8]{inputenc}
%\usepackage[latin1]{inputenc}

%% English version

%% MIEIC options
\usepackage[mieic]{feupteses}
%\usepackage[mieic,juri]{feupteses}
%\usepackage[mieic,final]{feupteses}
%\usepackage[mieic,final,onpaper]{feupteses}

%% Additional options for feupteses.sty: 
%% - onpaper: links are not shown (for paper versions)
%% - backrefs: include back references from bibliography to citation place

%% Uncomment the next lines if side by side graphics used
%\usepackage[lofdepth,lotdepth]{subfig}
%\usepackage{graphicx}
%\usepackage{float}

%% Include color package
\usepackage{color}
\definecolor{cloudwhite}{cmyk}{0,0,0,0.025}

%% Include source-code listings package
\usepackage{listings}
\lstset{ %
 language=C,                        % choose the language of the code
 basicstyle=\footnotesize\ttfamily,
 keywordstyle=\bfseries,
 numbers=left,                      % where to put the line-numbers
 numberstyle=\scriptsize\texttt,    % the size of the fonts that are used for the line-numbers
 stepnumber=1,                      % the step between two line-numbers. If it's 1 each line will be numbered
 numbersep=8pt,                     % how far the line-numbers are from the code
 frame=tb,
 float=htb,
 aboveskip=8mm,
 belowskip=4mm,
 backgroundcolor=\color{cloudwhite},
 showspaces=false,                  % show spaces adding particular underscores
 showstringspaces=false,            % underline spaces within strings
 showtabs=false,                    % show tabs within strings adding particular underscores
 tabsize=2,	                    % sets default tabsize to 2 spaces
 captionpos=b,                      % sets the caption-position to bottom
 breaklines=true,                   % sets automatic line breaking
 breakatwhitespace=false,           % sets if automatic breaks should only happen at whitespace
 escapeinside={\%*}{*)},            % if you want to add a comment within your code
 morekeywords={*,var,template,new}  % if you want to add more keywords to the set
}

%% Uncomment to create an index (at the end of the document)
%\makeindex

%% Path to the figures directory
%% TIP: use folder ``figures'' to keep all your figures
\graphicspath{{figures/}}

%%----------------------------------------
%% TIP: if you want to define more macros, use an external file to keep them
%some macro definitions

% format
\newcommand{\class}[1]{{\normalfont\slshape #1\/}}

% entities
\newcommand{\Feup}{Faculdade de Engenharia da Universidade do Porto}

\newcommand{\svg}{\class{SVG}}
\newcommand{\scada}{\class{SCADA}}
\newcommand{\scadadms}{\class{SCADA/DMS}}

%%----------------------------------------

%%========================================
%% Start of document
%%========================================
\begin{document}

%%----------------------------------------
%% Information about the work
%%----------------------------------------
\title{Create Your Own Future: Forecast Complex Behavior in Digital Marketing}
\author{Pedro Manuel Santos Borges}

%% Uncomment next line for date of submission
%\thesisdate{July 31, 2008}

%%Uncomment next line for copyright text if used
%\copyrightnotice{Name of the Author, 2008}

\supervisor{Supervisor}{João Mendes Moreira (PhD) - FEUP}

%% Uncomment next line if necessary
\supervisor{Co-Supervisor}{Hugo Sereno Ferreira (PhD) - FEUP}
\supervisor{Company Supervisor}{João Azevedo - ShiftForward}

%% Uncomment committee stuff in the final version if used
%\committeetext{Approved in oral examination by the committee:}
%\committeemember{Chair}{Doctor Name of the President}
%\committeemember{External Examiner}{Doctor Name of the Examiner}
%\committeemember{Supervisor}{Doctor Name of the Supervisor}
%\signature


%% Specify cover logo (in folder ``figures'')
\logo{uporto-feup.pdf}

%% Uncomment next line for additional text  below the author's name (front page)
\additionalfronttext{Technical Report}

%%----------------------------------------
%% Preliminary materials
%%----------------------------------------

% remove unnecssary \include{} commands
\begin{Prolog}
  \chapter*{Abstract}

The online advertisement industry handles a large quantity of money and users
everyday.
This industry is always trying to get more efficient, for example, by enhancing
the targeting of online advertising campaigns. 

This pursuit of efficiency on the world of online advertising turned simpler
methods of prediction unable to report an accurate number of
impressions, used to calculate the value
of a publisher's inventory. The introduction of concepts like frequency capping made that very clear.

There is now the necessity not only to predict the number of visits, but also to
predict when this visits will happen, what the user did before going to that
website and who he is.

In this document that concept will be approached using Data Mining techniques, such as classification and clustering, in order to generate a future ad request log
using only past data.

This generated results will be perfect afterwards, to be used on simulators
capable of calculate important metrics, for publishers and advertisers, for a set
of campaigns.

\chapter*{Resumo}

O mercado da publicidade online envolve diariamente muito dinheiro e muitos utilizadores. 
Este mercado que esta constantemente à procura de formas de se tornar
mais eficiente, por exemplo, melhorando o público alvo das suas campanhas
publicitárias. 

Esta busca pela eficiencia no mercado da publicidade online tornou metodos de
previsão mais simples incapazes de calcular correctamente o número de
impressões, utilizadas para calcular o valor do inventario de um
\textit{publisher}.
A introdução de conceitos como o \textit{frequency capping} torna isso muito evidente.

Há actualmente a necessidade de não só prever o número de visitas, como também
quando vão ocurrer essas visitas, o que o utilizador fez antes de lá chegar e
quem é
o utilizador em questão. 

Neste trabalho esse conceito vai ser abordado recorrendo a técnicas de
\textit{Data Mining}, como a classificação e o \textit{clustering}, de forma a conseguir gerar
um registo futuro de pedidos de publicidade utilizando apenas dados passados.

Os registos gerados estarão prontos a serem posteriormente utilizados, em
simuladores capazes de calcular os resultados, para um universo de campanhas.
 % the abstract
  %\chapter*{Acknowledgements}

Aliquam id dui. Nulla facilisi. Nullam ligula nunc, viverra a, iaculis
at, faucibus quis, sapien. Cum sociis natoque penatibus et magnis dis
parturient montes, nascetur ridiculus mus. Curabitur magna ligula,
ornare luctus, aliquam non, aliquet at, tortor. Donec iaculis nulla
sed eros. Sed felis. Nam lobortis libero. Pellentesque
odio. Suspendisse potenti. Morbi imperdiet rhoncus magna. Morbi
vestibulum interdum turpis. Pellentesque varius. Morbi nulla urna,
euismod in, molestie ac, placerat in, orci. 

Ut convallis. Suspendisse luctus pharetra sem. Sed sit amet mi in diam
luctus suscipit. Nulla facilisi. Integer commodo, turpis et semper
auctor, nisl ligula vestibulum erat, sed tempor lacus nibh at
turpis. Quisque vestibulum pulvinar justo. Class aptent taciti
sociosqu ad litora torquent per conubia nostra, per inceptos
himenaeos. Nam sed tellus vel tortor hendrerit pulvinar. Phasellus
eleifend, augue at mattis tincidunt, lorem lorem sodales arcu, id
volutpat risus est id neque. Phasellus egestas ante. Nam porttitor
justo sit amet urna. Suspendisse ligula nunc, mollis ac, elementum
non, venenatis ut, mauris. Mauris augue risus, tempus scelerisque,
rutrum quis, hendrerit at, nunc. Nulla posuere porta orci. Nulla dui. 

Fusce gravida placerat sem. Aenean ipsum diam, pharetra vitae, ornare
et, semper sit amet, nibh. Nam id tellus. Etiam ultrices. Praesent
gravida. Aliquam nec sapien. Morbi sagittis vulputate dolor. Donec
sapien lorem, laoreet egestas, pellentesque euismod, porta at,
sapien. Integer vitae lacus id dui convallis blandit. Mauris non
sem. Integer in velit eget lorem scelerisque vehicula. Etiam tincidunt
turpis ac nunc. Pellentesque a justo. Mauris faucibus quam id
eros. Cras pharetra. Fusce rutrum vulputate lorem. Cras pretium magna
in nisl. Integer ornare dui non pede. 

\vspace{10mm}
\flushleft{The Name of the Author}
  % the acknowledgments
  %\cleardoublepage
\thispagestyle{plain}

\vspace*{8cm}

\begin{flushright}
   \textsl{``You should be glad that bridge fell down. \\
           I was planning to build thirteen more to that same design''} \\
\vspace*{1.5cm}
           Isambard Kingdom Brunel
\end{flushright}
       % initial quotation if desired
  \cleardoublepage
  \pdfbookmark[0]{Table of Contents}{contents}
  \tableofcontents
  \cleardoublepage
  \pdfbookmark[0]{List of Figures}{figures}
  \listoffigures
  \cleardoublepage
  \pdfbookmark[0]{List of Tables}{tables}
  \listoftables
  \chapter*{Abbreviations}
\chaptermark{ABBREVIATIONS}

\begin{flushleft}
\begin{tabular}{l p{0.8\linewidth}}
ad    & advertising
adtech   & Online Marketing\\
CPA   & Cost-per-Action\\
CPC   & Cost-per-Click\\
CPL   & Cost-per-Lead\\
CPO   & Cost-per-Order\\
CPM   & Cost-per-Mile\\
DSP   & Demand Side Platform\\
RTB   & Real Time Bidding\\
SSP   & Supply Side Platform\\
\end{tabular}
\end{flushleft}

  % the list of abbreviations used
\end{Prolog}

%%----------------------------------------
%% Body
%%----------------------------------------
\StartBody

%% TIP: use a separate file for each chapter
\chapter{Introduction} \label{chap:intro}


\section{Context and Framing} \label{sec:context}

The online marketing is a growing multibillion-dollar industry \cite{PricewaterhouseCoopers2013}
which is expected to continue its fast growth.\cite{PricewaterhouseCoopers2013a}

This industry is always trying to get more efficient by getting more profit from assets it already own. 
Web users are the major assets of this industry, it makes money by exploiting the user behavior and characteristics, to target them with the
perfect campaign. Each campaign has its own target parameters, which limit the target user universe.
Online marketing industry core business is centered in web users, this industry has recorded almost every footprint each user makes on the web.
Future footprints of the web users allow to measure the behavior of a upcoming campaign, with this data it is possible to make a campaign more
effective and more profitable. Using future user data the adtech industry is able to fine tune its campaigns. 

\section{Project} \label{sec:proj}

Na continuação da secção anterior, e apenas no caso de ser um Projeto
e não uma Dissertação, esta secção apresenta resumidamente o projeto.

Nulla nec eros et pede vehicula aliquam. Aenean sodales pede vel
ante. Fusce sollicitudin sodales lacus. Maecenas justo mauris,
adipiscing vitae, ornare quis, convallis nec, eros. Etiam laoreet
venenatis ipsum. In tellus odio, eleifend ac, ultrices vel, lobortis
sed, nibh. Fusce nunc augue, dictum non, pulvinar sed, consectetuer
eu, ipsum. Vivamus nec pede. Pellentesque pulvinar fringilla dolor. In
sit amet pede. Proin orci justo, semper vel, vulputate quis, convallis
ac, nulla. Nulla at justo. Mauris feugiat dolor. Etiam posuere
fermentum eros. Morbi nisl ipsum, tempus id, ornare quis, mattis id,
dolor. Aenean molestie metus suscipit dolor. Aliquam id lectus sed
nisl lobortis rhoncus. Curabitur vitae diam sed sem aliquet
tempus. Sed scelerisque nisi nec sem. 

\section{Motivation and Goals} \label{sec:goals}

In the last few years, the online marketing campaigns have been evolving in such a way that today, these campaigns have a very well defined target.
In the past to predict the impact of a campaign metrics like unique users and total number of access to the work were enough but, with this more precise pin point 
of the target population the results given by the classical approach were useless.
Nowadays, some online ads can only be imprinted if a set of very specific requirements has been fulfilled, for example,
the users had to visit an e-commerce site in the last 24 hours. This brings causality into the equation, creating a new paradigm that makes 
the more traditional methods of prediction ineffective. To solve this problem, it's needed to predict the complete future data, so this generated data
can be used in simulations and the online campaigns can run on top of the future population.

The objective of this thesis is to develop a library capable of generating future access logs using past data from the same network.
The generated dataset (access log) must have the same attributes as the original.

This problem isn't trivial to solve and to better address it can be decomposed in smaller and simpler problems, such as:
\begin{itemize}
    \item Predict individuals of the original population which will reappear in the future. Future reappearances need to be completely characterized, including 
      all the parameters that compose an entry in the dataset, but transposed to the future. The value of each parameter must be valid an obey to the parameter
      rules, these rules are not hard-coded or explained at the beginning and must be inferred from the original data set. Solving this sub-problem also implies the identification
      of users who won't appear in future entries. Recurrence of users will be the first sub-problem to be addressed in this dissertation work. 
      Clustering algorithms will be explored to predict churn users and instance-based algorithms to select future entries. A more detailed information
      on this problem will be presented on chapter~\ref{chap:sota} in section -------TO BE COMPLETED-------.
    \item Predict new instances which had never appeared in the past entries of the dataset.
      These instances will represent new user entries, to predict the profile of each user classification algorithms will be used, to select 
      which past entries are relevant. Further information about these methods will be presented in
      chapter~\ref{chap:sota} in section -------TO COMPLETE-------.
\end{itemize}

\section{Report Structure} \label{sec:struct}

Besides this first introdutory chapter this report is divided in three additional chapters.
In Chapter~\ref{chap:sota}, a some basics of about online marketing are explained. In addition, it is also detailed the state of the art of
predicting a traffic on networks, data mining algorithms that can help this kind of problems. Lastly, there is
a small conclusion of which of those methods seem more fit to help solve this problem.
Chapter~\ref{chap:chap3} focus is the main steps that needed to be taken during the development of this thesis. A schedule of the work is also
presented on this chapter.
Chapter~\ref{chap:chap4} sums up the report giving a better context of all the review done in the final project.
 
\chapter{State of the Art} \label{chap:sota}

\section*{}

This chapter starts with an overview of online marketing in the last few years, followed by a review of
what was done and is been done in the field of predicting the traffic of a network.
Lastly, it is done a review of data mining algorithms that can be used to solve the same kind of problems as this thesis.

\section{Online Advertising Overview}

Before entering in details about the state of the art of the technologies that can be used to
help solve the presented problem, it is better to explain some basic concepts about the world of online advertising.

All advertising has the main purpose of getting a message to the people that will impact or influence them in some way,
therefore the same goal is applied to online advertising.
One of the metrics of advertising are impressions, which correspond to the number of times a user sees the message (the ad).\cite{kOA}
\textbf{Ads} can present itself in various sizes\cite{kOA2}, forms and locations \cite{kOA3}, and these characteristics are chosen both by the advertiser
and the publisher to better serve their purpose.
\textbf{Campaigns} are composed by two big parts, which are the ads that compose it and the target population that they pretend to reach,
including the rules of this targeting. For example, \textbf{frequency capping} to limit the number of times the same advertising is shown to the user \cite{kOA},
avoiding, in this way, showing the same ad multiple times in a row to the same user, that can lead to a bad response from his part.\cite{Buchbinder20141}

Nowadays, the main pricing models of online advertising are:
\begin{itemize}
\item\textbf{Cost-per-Mile} where the advertiser pays per impression. The main problem of this model is the advertiser as to pay to the publisher even
if the ad doesn't lead to any profit.
\item\textbf{Cost-per-Click} where the advertiser pays per click to the publisher. This model is more expensive per unit\cite{Performics}, but on overall can be more
profitable\cite{Performics} if the audience of the websites where the ad is imprinted is more interested in that kind of product/service\cite{Andrea2004}.
\item\textbf{Cost-per-lead} where the advertiser pays for a lead. If this model is being used the advertiser doesn't pay per number of impressions nor per clicks. Instead, pays only
if he gets valid information about the user, like the information of a sign up form for a community.
\item\textbf{Cost-per-Action} or \textbf{Cost-per-Order} where the advertiser is charged per buy or action. This model is similar to CPL but has in mind an instantaneous return of
the investment.
\end{itemize}

Traditionally, publishers sell their space to advertisers in bulk (\textbf{Ad networks}) this method has its \textit{ups} and \textit{downs}.
The obvious \textit{up} is that sometimes the advertiser gets premium spots at low prices.
On the other hand, one of the biggest drawbacks is that when the advertiser buys the impressions as a
closed package, sometimes impressions are not maximized in terms of profit.
Other problem of traditional methods that, although the \emph{CPA} and \emph{CPL} pricing methods minimize the risk for the advertiser, the responsibility of
optimizing conversion rate\footnote{See, e.g., \url{http://www.marketingterms.com/dictionary/conversion_rate/}} is still
on the ad network hands.\cite{Yuan:2013:RBO:2501040.2501980}

In the past few years, a new model called \textbf{Real Time Bidding}\index{Real Time Bidding}\index{RTB} has been gaining terrain \cite{Adfonic}. 
There are three main players in the world of \emph{RTB}\index{Real Time Bidding}:
\begin{itemize}
\item The\textbf{Demand Side Platform}\index{DSP} is a tool used by the advertisers to act on their behalf on the \emph{RTB}. \emph{DSPs}\index{DSP} allows them to set
their campaigns' parameters and to monitor the performance of the campaign. This way the advertisers try to get the best performance of their campaigns because 
\emph{DSPs} use algorithms driven by performance data.\cite{Gern201230}
\item The\textbf{Publisher} provides the inventory, that is comprised by accesses made by users. In some cases, the publisher uses \textbf{Supply Side Platforms}.
\emph{SSPs} help the publisher to better manage his inventory, and even let him set a reserve price for their inventory.\cite{Yuan:2013:RBO:2501040.2501980}
\item The\textbf{Ad Exchange} looks a little like a stock exchange, but in reality is a software platform that mediates the exchange. This exchange takes place in
a few milliseconds while the page loads.
\end{itemize}

\emph{RTB} allows some features of paid search advertising everywhere \cite{Gern201230}, because it allows the advertiser to better select
the inventory\footnote{this inventory is made of user accesses} where he wants their campaigns to run on.
The flexibility that \emph{RTB} gives to all the intervinients of this exchange is what demands the necessity of predicting the future inventory, to
better access its value.


\section{Network Traffic Prediction}\label{sec:network}

\section{Data Mining}\label{sec:datamining}

\subsection{Instance-based regression algorithms}\label{sec:instance}

\subsection{Classification Algorithms}\label{sec:classification}

\subsection{Clustering Algorithms}\label{sec:clust}

\subsection{Regression Algorithms}\label{sec:regr}

\section{Time Series}\label{sec:timeseries}

\chapter{Work Plan}\label{chap:chap3}

\section*{}

The current chapter comprehends the work plan for the dissertation, including its activities and respective time frames. It will also be explained the datasets that will be used in the project.

\section{Planning}

Given that the preparation phase is already finished, there will be approximately X MONTHS/WEEKS for the realization of the dissertation. That corresponds to a time frame between DATA X até DATA X. In order to complete the dissertation in the given time, it will be given more emphasis to the 

The planning is divided in the four following stages:
- Analysis and choice of the best approaches to the problem regarding what are the best data mining methods to test for a sample of results for the given problem.
- Testing of the chosen approaches in order to indentify which ones could possibly give the best results.
- Application of the best approach identified in the tests.
- Writing of the dissertation.

%\begin{figure}[!ht]
  %\begin{center}

    %\begin{ganttchart}[
      %y unit title=0.6cm,
      %y unit chart=0.7cm,
      %x unit=0.8cm,
      %vgrid,hgrid,
      %title height=1,
      %bar/.style={fill=gray!50},
      %progress label text={},
      %bar height=0.4]{12}

      %% Time labels
      %\gantttitle{2014}{12} \\
      %\gantttitle{February}{2}
      %\gantttitle{March}{2}
      %\gantttitle{April}{2}
      %\gantttitle{May}{2}
      %\gantttitle{June}{2}
      %\gantttitle{July}{2} \\

      %% Tasks
      %\ganttbar{Information system development}{2}{4} \\
      %\ganttbar{Assembly pipeline development}{4}{5} \\
      %\ganttbar{Web platform development}{5}{6} \\
      %\ganttbar{Transcriptome assembly}{7}{9} \\
      %\ganttbar{Transcriptome analysis}{8}{10} \\
      %\ganttbar{Thesis writing}{9}{11}
      %\ganttbar[bar/.style={fill=gray!20}]{}{2}{8}

    %\end{ganttchart}
  %\end{center}
  %\caption[Work distribution planning]{Work distribution planning.}
  %\label{fig:gantt}
%\end{figure}
%PLANEAR COMO DISTRIBUIR AS FASES e pôr mais lodo em cada uma

\section{Experimental Data}

%FALAR SOBRE OS DATASETS USADOS -> qual o tipo, o que tem, características

\section{Thesis Work Evaluation}

%DIZER COMO É QUE SE VAI AVALIAR A TESE -> presumo que seja o comparar os dados gerados com os dados futuros reais


\chapter{Conclusions}\label{chap:chap4}

\section*{}

This chapter sums up the objectives and restates the context of the project, along with its planned activities, and gives, as a final recap, its future work.

\section{Objectives}

As previous stated, the main objective of this dissertation is to develop a library capable of predicting future access logs based on past access logs,
where the generated access logs must have the same attributes as the original ones. This objective can be divided into two main problems, namely the
prediction of the individuals of the original dataset that will reappear in the future and the prediction of new instances which had never appeared
in the past entries of the dataset. 

\section{Project}

The project be the final product resulting of the fulfilment of what was previously stated. When the project is done, it will be possible to use the 
generated data to calculate the value of an inventory on future campaigns.

\section{Future Work}

The results will be evaluated and validated. Depending on the results, some features can be better explored to try
to capture certain behaviours that may not be captured by the developed work.


%%----------------------------------------
%% Final materials
%%----------------------------------------

%% Bibliography
%% Comment the next command if BibTeX file not used
%% bibliography is in ``myrefs.bib''
\PrintBib{myrefs}

%% comment next 2 commands if numbered appendices are not used
%\appendix
%\chapter{Loren Ipsum} \label{ap1:loren}

Depois das conclusões e antes das referências bibliográficas,
apresenta-se neste anexo numerado o texto usado para preencher a
dissertação.

\section{O que é o \emph{Loren Ipsum}?}

\emph{\textbf{Lorem Ipsum}} is simply dummy text of the printing and
typesetting industry. Lorem Ipsum has been the industry's standard
dummy text ever since the 1500s, when an unknown printer took a galley
of type and scrambled it to make a type specimen book. It has survived
not only five centuries, but also the leap into electronic
typesetting, remaining essentially unchanged. It was popularised in
the 1960s with the release of Letraset sheets containing Lorem Ipsum
passages, and more recently with desktop publishing software like
Aldus PageMaker including versions of Lorem Ipsum~\citep{kn:Lip08}. 

\section{De onde Vem o Loren?}

Contrary to popular belief, Lorem Ipsum is not simply random text. It
has roots in a piece of classical Latin literature from 45 BC, making
it over 2000 years old. Richard McClintock, a Latin professor at
Hampden-Sydney College in Virginia, looked up one of the more obscure
Latin words, consectetur, from a Lorem Ipsum passage, and going
through the cites of the word in classical literature, discovered the
undoubtable source. Lorem Ipsum comes from sections 1.10.32 and
1.10.33 of ``de Finibus Bonorum et Malorum'' (The Extremes of Good and
Evil) by Cicero, written in 45 BC. This book is a treatise on the
theory of ethics, very popular during the Renaissance. The first line
of Lorem Ipsum, ``Lorem ipsum dolor sit amet\ldots'', comes from a line in
section 1.10.32.

The standard chunk of Lorem Ipsum used since the 1500s is reproduced
below for those interested. Sections 1.10.32 and 1.10.33 from ``de
Finibus Bonorum et Malorum'' by Cicero are also reproduced in their
exact original form, accompanied by English versions from the 1914
translation by H. Rackham.

\section{Porque se usa o Loren?}

It is a long established fact that a reader will be distracted by the
readable content of a page when looking at its layout. The point of
using Lorem Ipsum is that it has a more-or-less normal distribution of
letters, as opposed to using ``Content here, content here'', making it
look like readable English. Many desktop publishing packages and web
page editors now use Lorem Ipsum as their default model text, and a
search for ``lorem ipsum'' will uncover many web sites still in their
infancy. Various versions have evolved over the years, sometimes by
accident, sometimes on purpose (injected humour and the like). 

\section{Onde se Podem Encontrar Exemplos?}

There are many variations of passages of Lorem Ipsum available, but
the majority have suffered alteration in some form, by injected
humour, or randomised words which don't look even slightly
believable. If you are going to use a passage of Lorem Ipsum, you need
to be sure there isn't anything embarrassing hidden in the middle of
text. All the Lorem Ipsum generators on the Internet tend to repeat
predefined chunks as necessary, making this the first true generator
on the Internet. It uses a dictionary of over 200 Latin words,
combined with a handful of model sentence structures, to generate
Lorem Ipsum which looks reasonable. The generated Lorem Ipsum is
therefore always free from repetition, injected humour, or
non-characteristic words etc. 


%% Index
%% Uncomment next command if index is required
%% don't forget to run ``makeindex pdis-en'' command
%\PrintIndex

\end{document}
