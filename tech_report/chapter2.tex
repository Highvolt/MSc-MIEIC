\chapter{State of the Art} \label{chap:sota}

\section*{}

This chapter starts with an overview of online marketing in the last few years.
Followed by a review of what was done and is been done in the field of predicting the traffic for a network.
Lastly, a review of data mining algorithms that can be used to solve the same kind of problems as this thesis.

\section{Online Advertising Overview}

Before enter in details on the state of the art of the technologies that can be used to help to solve the problem, it is better to explain
some basic concepts about the world of online advertising. 

All advertising has only one purpose getting their message to the people, in online advertising the same rule apply. \textbf{Impressions} are how
many times a user sees this message (the ad).\cite{kOA} \textbf{Ads} can present it self in various forms and locations \cite{kOA2}, this forms and appearances
\cite{kOA3} are selected by the advertisers to better serve its purpose. 
\textbf{Campaigns} are composed by two big parts the ads that compose it and the target population that they pretend to reach, including the rules of this targeting.
For example \textbf{frequency capping}, to limit the number of times the same advertising is shown to the user \cite{kOA}.

Nowadays, the main pricing models of online advertising are:
\begin{itemize}
\item\textbf{CPM} the advertiser pays per impression. The main problem of this model is the advertiser as to pay to the publisher even if the ad doesn't lead to any profit.
\item\textbf{CPC} the advertiser pays per click to the publisher. This model is more expensive per unit\cite{Performics}, but on overall can me more
profitable\cite{Performics} if the audience of the websites were the ad is imprinted is more interested in that kind of product/service\cite{Andrea2004}.
\end{itemize}


\section{Network Traffic Prediction}\label{sec:network}

\section{Data Mining}\label{sec:datamining}

\subsection{Instance-based regression algorithms}\label{sec:instance}

\subsection{Classification Algorithms}\label{sec:classification}

\subsection{Clustering Algorithms}\label{sec:clust}

\subsection{Regression Algorithms}\label{sec:regr}

\section{Time Series}\label{sec:timeseries}
