\chapter{State of the Art} \label{chap:sota}

\section*{}

This chapter starts with an overview of online marketing in the last few years, followed by a review of
what was done and is been done in the field of predicting the traffic of a network.
Lastly, it is done a review of data mining algorithms that can be used to solve the same kind of problems as this thesis.

\section{Online Advertising Overview}

Before entering in details about the state of the art of the technologies that can be used to
help to solve the presented problem, it is better to explain some basic concepts about the world of online advertising.

All advertising has the main purpose of getting a message to the people that will impact or influence them in some way,
therefore the same goal is applied to online advertising.
One of the metrics of advertising are impressions, which correspond to the number of times a user sees the message (the ad).\cite{kOA}
\textbf{Ads} can present itself in various size\cite{kOA2}, forms and locations \cite{kOA3}, these characteristics are chosen both by the advertiser
and the publisher to better serve its own purpose.
\textbf{Campaigns} are composed by two big parts, which are the ads that compose it and the target population that they pretend to reach,
including the rules of this targeting. For example, \textbf{frequency capping} to limit the number of times the same advertising is shown to the user \cite{kOA}.

Nowadays, the main pricing models of online advertising are:
\begin{itemize}
\item\textbf{CPM} where the advertiser pays per impression. The main problem of this model is the advertiser as to pay to the publisher even
if the ad doesn't lead to any profit.
\item\textbf{CPC} where the advertiser pays per click to the publisher. This model is more expensive per unit\cite{Performics}, but on overall can me more
profitable\cite{Performics} if the audience of the websites were the ad is imprinted is more interested in that kind of product/service\cite{Andrea2004}.
\item\textbf{CPL} where the advertiser pays for a lead. If this model is being used the advertiser doesn't not pay per number of impressions nor per clicks,
only pays if he gets valid information of the user, like filling a sign up form for a community.
\item\textbf{CPA} or \textbf{CPO} where the advertiser is charged per buy or action. This model is similar to CPL but has in mind an instantaneous return of
the investment.
\end{itemize}


\section{Network Traffic Prediction}\label{sec:network}

\section{Data Mining}\label{sec:datamining}

\subsection{Instance-based regression algorithms}\label{sec:instance}

\subsection{Classification Algorithms}\label{sec:classification}

\subsection{Clustering Algorithms}\label{sec:clust}

\subsection{Regression Algorithms}\label{sec:regr}

\section{Time Series}\label{sec:timeseries}
