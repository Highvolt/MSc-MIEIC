\chapter{Work Plan}\label{chap:chap3}

\section*{}

The current chapter comprehends the work plan for the dissertation, including its activities and respective time frames. It will also be explained the datasets that will be used in the project.

\section{Planning}

Given that the preparation phase is already finished, there will be approximately X MONTHS/WEEKS for the realization of the dissertation. That corresponds to a time frame between DATA X até DATA X. In order to complete the dissertation in the given time, it will be given more emphasis to the 

The planning is divided in the four following stages:
- Analysis and choice of the best approaches to the problem regarding what are the best data mining methods to test for a sample of results for the given problem.
- Testing of the chosen approaches in order to indentify which ones could possibly give the best results.
- Application of the best approach identified in the tests.
- Writing of the dissertation.

%\begin{figure}[!ht]
  %\begin{center}

    %\begin{ganttchart}[
      %y unit title=0.6cm,
      %y unit chart=0.7cm,
      %x unit=0.8cm,
      %vgrid,hgrid,
      %title height=1,
      %bar/.style={fill=gray!50},
      %progress label text={},
      %bar height=0.4]{12}

      %% Time labels
      %\gantttitle{2014}{12} \\
      %\gantttitle{February}{2}
      %\gantttitle{March}{2}
      %\gantttitle{April}{2}
      %\gantttitle{May}{2}
      %\gantttitle{June}{2}
      %\gantttitle{July}{2} \\

      %% Tasks
      %\ganttbar{Information system development}{2}{4} \\
      %\ganttbar{Assembly pipeline development}{4}{5} \\
      %\ganttbar{Web platform development}{5}{6} \\
      %\ganttbar{Transcriptome assembly}{7}{9} \\
      %\ganttbar{Transcriptome analysis}{8}{10} \\
      %\ganttbar{Thesis writing}{9}{11}
      %\ganttbar[bar/.style={fill=gray!20}]{}{2}{8}

    %\end{ganttchart}
  %\end{center}
  %\caption[Work distribution planning]{Work distribution planning.}
  %\label{fig:gantt}
%\end{figure}
%PLANEAR COMO DISTRIBUIR AS FASES e pôr mais lodo em cada uma

\section{Experimental Data}

%FALAR SOBRE OS DATASETS USADOS -> qual o tipo, o que tem, características

\section{Thesis Work Evaluation}

%DIZER COMO É QUE SE VAI AVALIAR A TESE -> presumo que seja o comparar os dados gerados com os dados futuros reais

