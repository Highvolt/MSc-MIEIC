\chapter{Work Plan}\label{chap:chap3}

\section*{}

The current chapter comprehends the work plan for the dissertation, including its activities and respective time frames.
It will also be explained the datasets that will be used in the project.

\section{Planning}

Given that the preparation phase is already finished, there will be approximately 5 months for the realization of the dissertation.
That corresponds to a time frame between February 2014 and July 2014.

From what has been previously stated, this problem is rather complex. In order to better
address it, it can be decomposed in smaller and simpler problems, such as:
\begin{itemize}
    \item Predict individuals of the original population which will reappear in the future. Future reappearances need to be completely characterized, including 
      all the parameters that compose an entry in the dataset, but transposed to the future. The value of each parameter must be valid and obey to the parameter
      rules, these rules are not hard-coded or explained at the beginning and must be inferred from the original data set. Solving this sub-problem also implies the identification
      of the users who won't appear in future entries. Recurrence of users will be the first sub-problem to be addressed in this dissertation work. 
      Clustering algorithms will be explored to predict churn users and instance-based algorithms to select future entries. A more detailed information
      on this problem was presented on chapter~\ref{chap:sota}.
    \item Predict new instances which had never appeared in the past entries of the dataset.
      These instances will represent new user entries and, to predict the profile of each user, it will be used classification algorithms, in order to select 
      which past entries are relevant. Further information about these methods
      was presented in
      chapter~\ref{chap:sota}.
\end{itemize}

In order to complete the dissertation in the given time, it will be given more emphasis to the
prediction of churn users and the prediction of users that continue in the
future, leaving the prediction of new instances for a later part of the
development of this dissertation.

\begin{figure}[h]
  \begin{center}

      \begin{ganttchart}[
      %hgrid,
    %vgrid,
    %x unit=4mm
      y unit title=0.6cm,
      y unit chart=1.5cm,
      x unit=0.8cm,
      vgrid,hgrid,
      title height=1,
      bar/.style={fill=gray!50,solid,draw=black},
      progress label text={},
      newline shortcut=true,
      bar label node/.append style={align=left},
      bar height=0.4
  ]{1}{12}
    % Time labels
      \gantttitle{2014}{12} \\
      \gantttitle{February}{2}
      \gantttitle{March}{2}
      \gantttitle{April}{2}
      \gantttitle{May}{2}
      \gantttitle{June}{2}
      \gantttitle{July}{2} \\

      % Tasks
      \ganttbar{Study, selection and testing of algorithms\ganttalignnewline
      to predict the users that maintain}{2}{5} \\
      \ganttbar{Study, selection and testing of algorithms\ganttalignnewline
      to predict the users that dropout}{5}{7} \\
      \ganttbar{Study, selection and testing of algorithms\ganttalignnewline
      to predict new users}{7}{8} \\
      \ganttbar{Writing of the dissertation}{8}{12} 
      \ganttbar[bar/.style={fill=gray!20}]{}{5}{7}
    \end{ganttchart}

  \end{center}
\caption{Gantt Chart}
\label{fig:gantt}
\end{figure}

The planning is divided in the four following stages, as we can see in figure~\ref{fig:gantt}:
\begin{enumerate}
  \item\label{itm:task1} Study, selection and testing of algorithms to predict the users that maintain.
  \item\label{itm:task2} Study, selection and testing of algorithms to predict the users that dropout.
  \item\label{itm:task3} Study, selection and testing of algorithms to predict new users.
  \item\label{itm:task4} Writing of the dissertation.
\end{enumerate}

The problem of predicting which users, present in the original dataset, will remain
in the future will be addressed in stage~\ref{itm:task1}. To complete the
information generated on this stage it is also needed to forecast the value of
each parameter that represents an entry on the ad request log file. The period
alloted for the completion of this stage is 8 weeks, from mid February until mid
April.

The next sub problem to be address it will be the churn prediction, which is the
identification of the users that were in the original dataset but will not be present on the generated
data set. This will be stage~\ref{itm:task2}i, which represents 6 weeks of
work, from beginning of April to mid May,
where the first 2 weeks are shared with the stage~\ref{itm:task1}, since the problem
addressed there also helps to solve this problem.

The stage~\ref{itm:task3} will focus on the generation of new users, to complete the
generated log file on the previous steps. This stage will last for about 4
weeks, starting in the first week of May and ending on first week of June, where the first 2
weeks will be shared with the task of finding users that dropout. The last 2
weeks will also be shared with stage~\ref{itm:task4}.

Stages \ref{itm:task1}, \ref{itm:task2} and \ref{itm:task3} also include test
and validation of the obtained results. In this stages, there will also take part
a better study and selection of the algorithms that are going to be used.

The longest and last stage, stage~\ref{itm:task4}, will consist on the writing of the
dissertation. For this part, the period alloted is 16 weeks, divided in two
phases, where the first 6 weeks (lighter area) will be only adding notes and little changes to the
dissertation and on the other 10 weeks (darker area) the work will be almost exclusively dedicated
to the writing of the final document.


\section{Experimental Data}

In order to get performance results of the different methods and models,
multiple datasets containing records of ad requests will be used. The parameters
present in the datasets will variate from one source to another. Normally all
the datasets will have per entry a timestamp, an user ID, a location ID and an
URL, plus an unknown number of additional parameters, such as cookies, etc.
Every dataset will be used separately.

\section{Thesis Work Evaluation}

Every proposed algorithm will be tested against multiple datasets. To each
algorithm, multiple performance measures will be made, such as accuracy, time taken,
accuracy/time, these results will be compared with other tested algorithms and
the results provided by ShiftForward of theirs algorithm.
