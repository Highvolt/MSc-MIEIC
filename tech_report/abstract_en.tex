\chapter*{Abstract}

The last changes that took place on the world of online advertising lead to an inefficiency of metrics, such as, number of impressions used to calculate the value
of a publisher's inventory. The appearance of concepts like frequency capping made that very clear.

There is now the necessity not only to predict the number of visits, but also when will the visits happen, what the user did before going to that
website and who he is.

In this document that concept will be approached using Data Mining techniques, such as, classification and clustering, in order to generate a future ad request log
using only past data.

\chapter*{Resumo}

As mudanças que aconteceram nos últimos tempos no mundo da publicidade online tornaram métricas como o número de impressões ineficientes para prever o valor de um
espaço publicitário. A introdução de conceitos como o \textit{frequency capping} torna isso muito evidente.

Há actualmente a necessidade de não só prever o número de visitas, como também quando vão ser essas visitas, o que é que o utilizador fez antes de lá chegar e qual
o utilizador em questão. 

Neste trabalho essa temática vai ser abordada recorrendo a técnicas de \textit{Data Mining} como a classificação e o \textit{clustering}, de forma a conseguir gerar
um futuro registo de pedidos de publicidade usando apenas dados passados.
