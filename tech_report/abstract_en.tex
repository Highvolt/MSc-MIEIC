\chapter*{Abstract}

Everyday a large quantity of money and uses moves through the online advertising
industry.
This industry is always trying to get more efficient. One of the examples of
this path towards efficiency is the more sharp targeting of online advertising
campaigns. 

These changes that took place on the world of online advertising lead
to an inefficiency of prediction methods of metrics, such as, number of
impressions, used to calculate the value
of a publisher's inventory. The introduction of concepts like frequency capping made that very clear.

There is now the necessity not only to predict the number of visits, but also when will the visits happen, what the user did before going to that
website and who he is.

In this document that concept will be approached using Data Mining techniques, such as, classification and clustering, in order to generate a future ad request log
using only past data.

This generated results will be perfect to afterwards, be used on simulators
capable of calculate important metrics, for publishers and advertisers for a set
of campaigns.

\chapter*{Resumo}

O mercado da publicidade online envolve diariamente muito dinheiro e muitos utilizadores. 
É também um mercado que esta constantemente à procura de formas de se tornar
mais eficiente.
Um dos exemplos desse caminho para o aumento da eficiencia é o targeting mais
especifico do público alvo das campanhas. 

Estas mudanças tornaram metodos de
previsão de métricas como o número de impressões ineficientes para prever o valor de um
espaço publicitário. A introdução de conceitos como o \textit{frequency capping} torna isso muito evidente.

Há actualmente a necessidade de não só prever o número de visitas, como também quando vão ser essas visitas, o que é que o utilizador fez antes de lá chegar e qual
o utilizador em questão. 

Neste trabalho essa temática vai ser abordada recorrendo a técnicas de \textit{Data Mining} como a classificação e o \textit{clustering}, de forma a conseguir gerar
um futuro registo de pedidos de publicidade usando apenas dados passados.

Os registos gerados estarão prontos a posteriormente serem utilizados em
simuladores capazes de calcular os resultados para um universo de campanhas.
