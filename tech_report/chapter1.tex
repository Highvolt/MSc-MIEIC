\chapter{Introduction} \label{chap:intro}


\section{Context and Framing} \label{sec:context}

The online marketing is a growing multibillion-dollar industry \cite{PricewaterhouseCoopers2013}
which is expected to continue its fast growth.\cite{PricewaterhouseCoopers2013a}

This industry is always trying to get more efficient by getting more profit from assets it already own. 
Web users are the major assets of this industry, it makes money by exploiting the user behavior and characteristics, to target them with the
perfect campaign. Each campaign has its own target parameters, which limit the target user universe.
Online marketing industry core business is centered in web users, this industry has recorded almost every footprint each user makes on the web.
Future footprints of the web users allow to measure the behavior of a upcoming campaign, with this data it is possible to make a campaign more
effective and more profitable. Using future user data the adtech industry is able to fine tune its campaigns. 

\section{Project} \label{sec:proj}

Na continuação da secção anterior, e apenas no caso de ser um Projeto
e não uma Dissertação, esta secção apresenta resumidamente o projeto.

Nulla nec eros et pede vehicula aliquam. Aenean sodales pede vel
ante. Fusce sollicitudin sodales lacus. Maecenas justo mauris,
adipiscing vitae, ornare quis, convallis nec, eros. Etiam laoreet
venenatis ipsum. In tellus odio, eleifend ac, ultrices vel, lobortis
sed, nibh. Fusce nunc augue, dictum non, pulvinar sed, consectetuer
eu, ipsum. Vivamus nec pede. Pellentesque pulvinar fringilla dolor. In
sit amet pede. Proin orci justo, semper vel, vulputate quis, convallis
ac, nulla. Nulla at justo. Mauris feugiat dolor. Etiam posuere
fermentum eros. Morbi nisl ipsum, tempus id, ornare quis, mattis id,
dolor. Aenean molestie metus suscipit dolor. Aliquam id lectus sed
nisl lobortis rhoncus. Curabitur vitae diam sed sem aliquet
tempus. Sed scelerisque nisi nec sem. 

\section{Motivation and Goals} \label{sec:goals}

In the last few years, the online marketing campaigns have been evolving in such a way that today, these campaigns have a very well defined target.
In the past to predict the impact of a campaign metrics like unique users and total number of access to the work were enough but, with this more precise pin point 
of the target population the results given by the classical approach were useless.

Nowadays, some online ads can only be imprinted if a set of very specific requirements has been fulfilled, for example,
the users had to visit an e-commerce site in the last 24 hours. This brings causality into the equation, creating a new paradigm that makes 
the more traditional methods of prediction ineffective. To solve this problem, it's needed to predict the complete future data, so this generated data
can be used in simulations and the online campaigns can run on top of the future population.

The objective of this thesis is to develop a library capable of generating future access logs using past data from the same network.
The generated dataset (access log) must have the same attributes as the original.

This problem isn't trivial to solve and to better address it can be decomposed in smaller and simpler problems, such as:
\begin{itemize}
    \item Predict individuals of the original population which will reappear in the future. Future reappearances need to be completely characterized, including 
      all the parameters that compose an entry in the dataset, but transposed to the future. The value of each parameter must be valid an obey to the parameter
      rules, these rules are not hard-coded or explained at the beginning and must be inferred from the original data set. Solving this sub-problem also implies the identification
      of users who won't appear in future entries. Recurrence of users will be the first sub-problem to be addressed in this dissertation work. 
      Clustering algorithms will be explored to predict churn users and instance-based algorithms to select future entries. A more detailed information
      on this problem will be presented on chapter~\ref{chap:sota} in section -------TO BE COMPLETED-------.
    \item Predict new instances which had never appeared in the past entries of the dataset.
      These instances will represent new user entries, to predict the profile of each user classification algorithms will be used, to select 
      which past entries are relevant. Further information about these methods will be presented in
      chapter~\ref{chap:sota} in section -------TO COMPLETE-------.
\end{itemize}

\section{Report Structure} \label{sec:struct}

Besides this first introductory chapter this report is divided in three additional chapters.
In Chapter~\ref{chap:sota}, a some basics of about online marketing are explained. In addition, it is also detailed the state of the art of
predicting a traffic on networks, data mining algorithms that can help this kind of problems. Lastly, there is
a small conclusion of which of those methods seem more fit to help solve this problem.
Chapter~\ref{chap:chap3} focus is the main steps that needed to be taken during the development of this thesis. A schedule of the work is also
presented on this chapter.
Chapter~\ref{chap:chap4} sums up the report giving a better context of all the review done in the final project.
