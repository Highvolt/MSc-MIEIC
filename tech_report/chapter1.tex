\chapter{Introduction} \label{chap:intro}


\section{Context and Framing} \label{sec:context}

The online marketing is a growing multibillion-dollar industry \cite{PricewaterhouseCoopers2013}
which is expected to continue its fast growth.\cite{PricewaterhouseCoopers2013a}

This industry is always trying to get more efficient by getting more profit from assets it already own. 
Web users are the major assets of this industry, which makes money by exploiting the user behavior and characteristics, in order to target them with the
perfect campaign. Each campaign has its own target parameters, which limit the target user universe.
Online marketing industry core business is centered in web users and this industry has recorded almost every footprint each user makes on the web.
Future footprints of the web users allows the measure of the behavior of a upcoming campaign and, with this data, it is possible to make a campaign more
effective and more profitable. Therefore, using future user data allows the adtech industry to be able to fine tune its campaigns. 
Campaigns are composed by parameter definitions and rules and can be defined as
queries over the ad requests data. So the ideal way to forecast the effectiveness
of a campaign is by running these queries over generated ad requests data. This
is why is so important to be able to generate future ad requests data.

The most common platforms that will benefit from this data are Custom-built Ad
Servers and \hyperref[itm:adex]{Exchanges}, \hyperref[itm:ssp]{Sell-side platforms} (SSPs) and
\hyperref[itm:dsp]{Demand-side platforms}
(DSPs). These platforms are further explored section~\ref{sec:adover}.


\section{Project} \label{sec:proj}

The main objective of this project is to develop a library capable of generating
future ad requests logs from recorded past requests.
To solve this problem multiple data mining classification and clustering
techniques will be explored in order to find the fastest and more accurate way
to do it. 
This generator will use real ad request logs from various sources in CSV files.
The generated files will need to respect the same structure as the given file.

The focus of this thesis is only one part of a more complex problem, the
forecast the behaviour of campaigns in the future. The resultant ad request log
will be validated using a simulator supplied by ShiftForward.


%The project consists of creating a library capable of predicting the future access data logs based only on data logs of previous months.
%The hearth of the project lies in applying various data mining techniques to identify what will be the behavior of the users on the future,
%based on their behavior on the past.

%There are three main components on this project
%\begin{itemize}
%\item \textbf{The original dataset}, which is a CSV file with registry of accesses for a given network. Usually, there is 
  %a time, an user ID, a location ID, a URL and some additional parameters (cookies,etc. which may or may not be present from dataset to dataset) per entry.
%\item \textbf{A library capable of generating real like data} logs of future data based only on the past data. This is the work that will be developed during
  %this dissertation. The generation of this data logs must rely on the utilization of data mining techniques.
%\item \textbf{A simulator} capable of running campaigns over the generated dataset, to give advertising performance metrics to compare with real data for validation.
%\end{itemize}

\section{Motivation and Goals} \label{sec:goals}

In the last few years, the online marketing has been getting more complex. In
such a way that, today campaigns have a very well defined target sometimes with
a few hundreds sets of rules and limitations. This poses a big problem to
simpler prediction models that normally doesn't predict all ad request
parameters, this way limiting the parameters where queries can be done.


Nowadays, some online ads can only be imprinted if a set of very specific requirements has been fulfilled, for example,
the users had to visit an e-commerce site in the last 24 hours. This brings causality into the equation, creating a new paradigm that makes 
the more traditional methods of prediction ineffective. To solve this problem
and to be able to get fast responses to complex queries of concurrent campaigns,
simulate the algorithms executed by ad servers of the client
and ultimately parallelize the computation of the results for the
queries,
predicting the complete future data. This generated data
can be used in simulations and the online campaigns can run on top of the future population.

The objective of this thesis is to develop a library capable of generating future access logs using past data from the same network.
The generated dataset (access log) must have the same attributes as the original.

This problem isn't trivial to solve and to better address it can be decomposed in smaller and simpler problems, such as:
\begin{itemize}
    \item Predict individuals of the original population which will reappear in the future. Future reappearances need to be completely characterized, including 
      all the parameters that compose an entry in the dataset, but transposed to the future. The value of each parameter must be valid and obey to the parameter
      rules, these rules are not hard-coded or explained at the beginning and must be inferred from the original data set. Solving this sub-problem also implies the identificationof users who won't appear in future entries. Recurrence of users will be the first sub-problem to be addressed in this dissertation work. 
      Clustering algorithms will be explored to predict churn users and instance-based algorithms to select future entries. A more detailed information
      on this problem will be presented on chapter~\ref{chap:sota}.
    \item Predict new instances which had never appeared in the past entries of the dataset.
      These instances will represent new user entries and, to predict the profile of each user, it will be used classification algorithms, in order to select 
      which past entries are relevant. Further information about these methods will be presented in
      chapter~\ref{chap:sota}.
\end{itemize}

\section{Report Structure} \label{sec:struct}

Besides this first introductory chapter, this report is divided in three additional chapters.
In Chapter~\ref{chap:sota} it is explained some basic knowledge about online marketing. In addiction, there is
a small conclusion of which of those methods seem more fit to help solve this problem.
Chapter~\ref{chap:chap3} focus on the main steps that need to be taken during the development of this thesis. A schedule of the work is also
presented on this chapter.
Chapter~\ref{chap:chap4} sums up the report, giving a better context of all the review done in the final project.
