\chapter{Introduction} \label{chap:intro}


\section{Context and Framing} \label{sec:context}

The online marketing is a growing multibillion-dollar industry \cite{PricewaterhouseCoopers2013}
which is expected to continue its fast growth.\cite{PricewaterhouseCoopers2013a}

This industry is always trying to get more efficient by getting more profit from assets it already own. 
Web users are the major assets of this industry, it makes money by exploiting the user behavior and characteristics, to target them with the
perfect campaign. Each campaign has its own target parameters, which limit the target user universe.
Online marketing industry core business is centered in web users, this industry has recorded almost every footprint each user makes on the web.
Future footprints of the web users allow to measure the behavior of a upcoming campaign, with this data it is possible to make a campaign more
effective and more profitable. Using future user data the adtech industry is able to fine tune its campaigns. 

\section{Project} \label{sec:proj}

Na continuação da secção anterior, e apenas no caso de ser um Projeto
e não uma Dissertação, esta secção apresenta resumidamente o projeto.

Nulla nec eros et pede vehicula aliquam. Aenean sodales pede vel
ante. Fusce sollicitudin sodales lacus. Maecenas justo mauris,
adipiscing vitae, ornare quis, convallis nec, eros. Etiam laoreet
venenatis ipsum. In tellus odio, eleifend ac, ultrices vel, lobortis
sed, nibh. Fusce nunc augue, dictum non, pulvinar sed, consectetuer
eu, ipsum. Vivamus nec pede. Pellentesque pulvinar fringilla dolor. In
sit amet pede. Proin orci justo, semper vel, vulputate quis, convallis
ac, nulla. Nulla at justo. Mauris feugiat dolor. Etiam posuere
fermentum eros. Morbi nisl ipsum, tempus id, ornare quis, mattis id,
dolor. Aenean molestie metus suscipit dolor. Aliquam id lectus sed
nisl lobortis rhoncus. Curabitur vitae diam sed sem aliquet
tempus. Sed scelerisque nisi nec sem. 

\section{Motivation and Goals} \label{sec:goals}

The ultimate trend on online marketing campaigns bring a whole new set of limitations to the target user universe. Nowadays, some online ads can only
be imprinted under a set of very specific requirements, for example, the users had to visit an e-commerce site in the last 24 hours. 
This brings causality into the equation, creating a new paradigm that makes 
the more traditional methods of prediction ineffective. To solve this problem, it's needed to predict the complete future data.

The objective of this thesis is to develop a library capable of generating future access logs using past data from the same network.
The generated dataset (access log) must have the same attributes as the original.

This problem isn't trivial to solve and to better address it can be decomposed in smaller and simpler problems, such as:
\begin{itemize}
    \item Predict users who will reappear in the future. Future reappearances need to be completely characterized, including 
      all the parameters that compose an entry in the dataset. Solving this sub-problem also implies the identification of users who
      won't appear in futures entries. Recurrence of users will be the first sub-problem to be addressed in this dissertation work. 
      Clustering algorithms will be explored to predict churn users and instance-based algorithms to select future entries. A more detailed information
      on this problem will be presented on chapter~\ref{chap:sota} in section -------TO BE COMPLETED-------.
    \item Generate users that never appeared in the past based on mean profiles of new entries in the past. The future dataset will never
      be complete without the data from step. In this part, classification algorithms will be used, further developments will be presented in
      chapter~\ref{chap:sota} in section -------TO COMPLETE-------.
\end{itemize}

\section{Report Structure} \label{sec:struct}

Para além da introdução, esta dissertação contém mais x capítulos.
No capítulo~\ref{chap:sota}, é descrito o estado da arte e são
apresentados trabalhos relacionados. 
%\todoline{Complete the document structure.}
No capítulo~\ref{chap:chap3}, ipsum dolor sit amet, consectetuer
adipiscing elit.
No capítulo~\ref{chap:chap4} praesent sit amet sem. 
No capítulo~\ref{chap:concl}  posuere, ante non tristique
consectetuer, dui elit scelerisque augue, eu vehicula nibh nisi ac
est. 
